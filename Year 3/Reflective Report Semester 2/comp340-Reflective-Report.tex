% Please do not change the document class
\documentclass{scrartcl}

% Please do not change these packages
\usepackage[hidelinks]{hyperref}
\usepackage[none]{hyphenat}
\usepackage{setspace}
\doublespace

% You may add additional packages here
\usepackage{amsmath}
\usepackage{graphicx}
\usepackage{wrapfig}
\graphicspath{ {./images/} }

% Please include a clear, concise, and descriptive title
\title{CPD Report} 

% Please do not change the subtitle
\subtitle{COMP330 CPD Report}

% Please put your student number in the author field
\author{1507516}

\begin{document}

\maketitle


\section{Career Goals}
In this final semester I have encountered a lot of new challenges compared to previous years, especially emotional and mental problems due to the pressure of final year, and the effect this semester has on my entire life.
My career goals for after university are to start seeking employment within the games industry, whether that be indie or AAA. If that fails, I may start looking for an IT tech job just to get some work experience, as they seem to be more common.
In terms of what programming roles I am interested in, I would like to persue a career in procedural generation and AI, along side generalist programming roles.

\section{Summary of Key Challenges}


%%%% Challenge 1 - the affective domain, such as your ability to identify and manage different mental states and emotions
%Feeling like I am not intellegent enough to understand the concepts behind something, and am unable to
\subsection{Not feeling intellegent enough to carry out complex tasks}
% Justification and importance
Throughout this semester I spent a lot of time working on tasks that I thought I was not intellegent enough to learn, just because there was a very steep learning curve associated to some of the projects I have been working on this semseter.
Learning how to aproach new complex tasks is important because in this industry, new complex software is being developed all the time, so being able to feel confident enough to start learning is an important skill.
\par

% How the issue affects my work
This issue effected both my dissertation and group game project, for my dissertation, there are a lot of very complex terms and AI technqiues that I felt overwhelmed by.

\par

% SMART Plan to avoid this in the future
I know that this issue is less to do with my intellegence and more to do with my attitude to learning, as I know if I put the time into learning something, I can do it.

When I was researching how to overcome this issue, I found a reddit post that many people in the comments stated that the best way to grasp concepts is to be exposed to them multiple times with space inbetweem them \cite{Reddit}.

So over this summer I will start researching more complex topics that I have felt to be too unintellegent for, such as design patterns, and complex C++ OOP designs, then sit down and read about these topics at least once a week so their concepts will eventually sink in, by the end of summer, I will reflect on my progress towards these topics by viewing my new coding styles for my game project that I am working on.



%%%% Challenge 2 - the interpersonal domain, such as your ability to communicate and organise activity with peers
% Ingaging in discussions more, instead of just listening
\subsection{Engaging more  in group discussions}

% Justification and importance
At my time in college and university I have always been more of a listener than a talker, and I often find it hard to engage in group discussions. This effects my learning because being able to engage in group discussions can lead to new ideas and perspectives, as well as being able to solve more complex problems that I would not be able to sole on my own.

% How the issue affects my work
Discussing complex dissertation topics, such as statistical analysis and the use of R within my year group could have given me better insights and understanding towards these complex topics.


% SMART Plan to avoid this in the future
When researching methods to overcome these issues, one source \cite{Group} suggests to use a methods of conversation that gathers viewpoints from everyone in the groip by using "Circle stories".
In future

\par






%%%% Challenge 3 - the dispositional domain, such as your ability to manage time and remain disciplined
% Stopping working for the day because I got something done.
\subsection{Avoid stopping a task because I think I have done enough for one day}

% Justification and importance
Throughout this year, when there are no looming deadlines, I tend to stop working on my uni work when I feel I have accomplished something, no matter how small.

% How the issue affects my work
This generally effects my work as when I get something done, it disrupts the flow of my work and takes a lot longer to finish a task. Moreover, it also means I tend to leave more work to do nearer the project deadline.

% SMART Plan to avoid this in the future
To overcome this it depends on what type of job I get after university, if I get a 9-5 job, then this will solve that problem, because I will not have the temptation to leave early. However if I get a freelance job etc, the motivation of getting paid should be enough to motivate me to spend more time working on the project.
\par



%%%% Challenge 4 - the cognitive domain, such as your knowledge of programming languages or frameworks
% using python at the end of this semester
\subsection{Using a wide range of programming languages}
% Justification and importance
In my final year, I have been switching between a wide verity of programming languages, these include:
Assembly, C++, JAVA, Python, R and C\#.

% How the issue affects my work

% SMART Plan to avoid this in the future

\par




%%%% Challenge 5 - the procedural domain, such as your abilities to apply computing concepts and problem solving skills
%
\subsection{ }

% Justification and importance

% How the issue affects my work


% SMART Plan to avoid this in the future

\par






%%% OTHER
% Getting used to GVG-AI framework
%

\bibliographystyle{ieeetr}
\bibliography{comp340-Reflective-Report}


\end{document}
