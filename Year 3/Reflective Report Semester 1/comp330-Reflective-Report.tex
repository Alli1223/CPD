% Please do not change the document class
\documentclass{scrartcl}

% Please do not change these packages
\usepackage[hidelinks]{hyperref}
\usepackage[none]{hyphenat}
\usepackage{setspace}
\doublespace

% You may add additional packages here
\usepackage{amsmath}
\usepackage{graphicx}
\usepackage{wrapfig}
\graphicspath{ {./images/} }

% Please include a clear, concise, and descriptive title
\title{Reflective Report} 

% Please do not change the subtitle
\subtitle{COMP330 Reflective Report}

% Please put your student number in the author field
\author{1507516}

\begin{document}

\maketitle


\section{Career Goals}
This is my final year studying at university, and a chance to put the skills I have learnt over the previous years to use. 

This report will cover the issues I have had over the first semester and propose SMART goals to help resolve the challenges I faced this year.
The key challenges raised below are based of the most common issues found in my weekly CPD report.
%SMART actions to resolve these issues


\section{Summary of Key Challenges}

%%%% Challenge 1

% Communication of what I am working on to group team when I am not present for scrum meetings
% Getting hung up on a task for a long time, and not switch to another one.
\subsection{Communication within a team} \label{Comm}

% Justification and importance
Communication within a team is critical to the success of a project, especially within the games and software development industry communication within large teams is vital.
Over this semester there were a few instances of where I would not keep the team in the loop as to what I was working on for their game.

\par

% How the issue affects my work
This lead to the team not thinking that I was doing any work for them, and also if I got stuck on something for a while, the team might have been able to help.
This also leads on to the challenge of leaving a problem I am struggling with, as mentioned in section \ref{Avoid} where i will not leave a feature until I complete it, even if I get blocked by something.
This affected my work and the work of my team, because if I had kept my team in the loop on what I was doing every day, they would know where I was at with each feature of the game.
\par

% SMART Plan to avoid this in the future

To avoid doing this next semester and throughout my computing career, I will aim to converse within the discord programming channel when I am blocked by a problem, to see if anyone has encountered a similar problem before and can help. I will measure this by looking at the conversation history within the programming discord channel throughout next semester, and see how well I was able to communicate my programming issues.




%%%% Challenge 2
\subsection{Public Speaking}

% Justification and importance
Being able to speak to a large audience is very important within games, and in any career. The art of being able to give a presentation or give public talks and enunciate clearly is a very useful skill to learn, it allows you to be able to give your point across clear and concisely. Furthermore it is required for more management type jobs.
I have found it very hard to stand up in front of even just a few people to give a presentation on something I know. I'm not sure why i've never been good at public speaking, but it is a skill I aim to learn over this course.


% How the issue affects my work
Not being able to give good presentations doesn't affect me much within this course, however when I leave to get a job it will affect me considerably.
When I have presented before, not only just this semester, I tend to just read from the slides, because I find that the stress of presenting in front of people blanks my mind from expanding on the comments.

% SMART Plan to avoid this in the future

There is another 15 minute presentation coming up very soon, in the next week, and I aim to prepare my self well to give a good, clear presentation for then.
To be able to avoid my problems before of just reading off the screen I will make cards to read off when I present, so I will not have to look at the screen. This should lead to a better flowing presentation. Further more I will try and incorporate some interactive elements to the presentation to keep it more entertaining.

\par






%%%% Challenge 3
\subsection{Avoid Getting Blocked or Stuck on a Feature} \label{Avoid}

% Justification and importance
During the course of the first semester there were a few times that I got stuck or blocked on adding a feature to the game, and instead of putting the feature back on the backlog, I would just wait until the the blocker cleared, or I found a solution to my problem.



% How the issue affects my work
This issue affected me and my team throughout the semester because it slowed the progress of the game and it also leads to bad communication as talked about in section \ref{Comm}.


% SMART Plan to avoid this in the future
To avoid this next semester, the first few weeks back I will keep on top of the scrum board and if I get stuck on a while for 2 times the estimated time of the task, I will put the card on the backlog and pick up a new card that is more manageable.

\par






%%%% Challenge 4
\subsection{Discipline}

% Justification and importance
Being able to focus on work and not get detracted by other, more appealing tasks is crucial to getting high quality work.

% How the issue affects my work
This affects my work as I tend to leave a lot of my work to the last minute and this leads to quite a bit of my work being rushed.  
This has been a re-occuring problem for the past few years.

% SMART Plan to avoid this in the future
To try and overcome this challenge, at the start of next semester I will aim to 

\par




%%%% Challenge 5
\subsection{Use of Source Control with a Team}

% Justification and importance

% How the issue affects my work

% SMART Plan to avoid this in the future

\par





% Trying to focus on the parts I like about projects to motivate myself to do them

% Proof reading




\end{document}
