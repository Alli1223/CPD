% Please do not change the document class
\documentclass{scrartcl}

% Please do not change these packages
\usepackage[hidelinks]{hyperref}
\usepackage[none]{hyphenat}
\usepackage{setspace}
\doublespace

% You may add additional packages here
\usepackage{amsmath}
\usepackage{graphicx}
\usepackage{wrapfig}
\graphicspath{ {./images/} }

% Please include a clear, concise, and descriptive title
\title{Reflective Report} 

% Please do not change the subtitle
\subtitle{COMP240 Reflective Report}

% Please put your student number in the author field
\author{1507516}

\begin{document}

\maketitle


\section{Career Goals}
Over the course of this year I have learned a lot of new skills, however I do feel more confident that I would like to pursue the career of game developer after university.
This report will cover the main issues I struggled with over this year.
%SMART actions to resolve these issues


\section{Summary of Key Issues}
%More reading of papers and books.
\subsection{Reading the Necessary Literature Before Solving A Complex Programming Problem}
I found my self a few times this year trying to tackle a problem such as networking or AI and finding that if I had read the necessary literature about the topic I was working on before hand, I would of found it far easier. For example when I was writing the research journals for the AI project I was doing a lot of reading in parallel to the writing, this meant I found a few clever solutions to a few of my problems I was struggling with in some of the academic literature. However for the networking module a few of the papers I read made the networking module a lot easier to understand what was going on when I was writing it, after I did the research journal.
To try and overcome this issue I will finish reading the books and academic literature I started this year over summer. I will also find some material about legacy systems and the other projects so I am more prepared for next year.

%Learning new game engine.
%Procedural domain
\subsection{Managing Project Tasks}
When managing tasks over the course of this year, I have primarily been using trello to keep track of the progress of the projects. However in almost all of my trello boards I have not followed the mountain goat user stories style. I'm pretty sure most of my trello boards contain some cards that are just tasks. I have found it hard to create cards that are not tasks. What I have found over the course of this semester is most of the time that I create a trello card is when I have either started working on something new or I have already done the task. This is not the proper way in which trello is suppose to be used.
In order to overcome this next year I will attempt to fully flesh out the trello baord at the beginning of the project instead of while I'm working on the project and get into the rhythm of picking a card to work on each day. Further more this will help with my even distribution of workload issue if I try and complete one trello card per day for each project.

%Even distribution of workload / not getting caught up in a project I like.
%Dispositional domain
\subsection{Even Distribution of Workload}
This has been a common issue for most of my life, and is going to be a hard issue to overcome. There are some projects that I really click with and will spend weeks on, and neglect other projects. This may be fine in game development, but in academia this means I will get a overall worse grade because the neglected projects will bring my grade down significantly.
In my weekly reports I have tried a few things to overcome this issue, but with no luck so far. However I have come up with an idea, because I sill am not too sure what I am going to do for my dissertation, over summer I will work on a game that covers most the disciplines that have been covered in the course so far, e.g. AI, PCG etc.. I will aim to work out what I enjoy the most and harness this flaw into a strength to work harder for my dissertation project.

%Not digging myself a hole in group projects / group project communication /Wokring with a large team
%Mental states and emotions (Active domain)
\subsection{Presentation Skills}
Over the course of this year, I have done a few presentations to the class, and some to all the years. I have always known that I am not great at public speaking but I feel now that I have done a few that there is a lot of room for improvement here. What I found when I did some of my presentations is that I would tend to just read the bullet points off the slide, instead of adding to the points to give the presentation some depth. There are also times where the pressure of doing the presentation will make me loose track of what I was saying, and my mind would go blank.
To try and overcome this problem, next year I will aim to create presentation notes for when I create the slides, I will then read off these notes when I do the presentation to keep myself focused on what the points in the slide are trying to portray. 

%Code
%Cognative Domain
\subsection{Writing Reusable Code}
Over the course I have written a lot of C++ code, and each time I write a bit of code I only think about how that bit of code can be used to solve that particular problem. So I will often use some functions that are only appropriate to that project, this means that when I start a new project I have to write \textit{similar} code to solve pretty much the same problem. This can be avoided if I can make some of my functions and classes more general so that other values can be passed in instead. An example of where I did this was in my AI project, the agentManager class I wrote to control the agents had to be re-written for when I copied the code into the networking project.
To attempt to overcome this issue, over the summer I will aim to refactor some of my existing non-reusable code into something that can be ported to other projects. E.g. A batch renderer in SDL that will render sprites that are set in different layers. This will help me understand the principles of how to make more reusable code.

%Presentation skills


\end{document}
